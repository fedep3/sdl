% THIS IS SIGPROC-SP.TEX - VERSION 3.1
% WORKS WITH V3.2SP OF ACM_PROC_ARTICLE-SP.CLS
% APRIL 2009
%
% It is an example file showing how to use the 'acm_proc_article-sp.cls' V3.2SP
% LaTeX2e document class file for Conference Proceedings submissions.
% ----------------------------------------------------------------------------------------------------------------
% This .tex file (and associated .cls V3.2SP) *DOES NOT* produce:
%       1) The Permission Statement
%       2) The Conference (location) Info information
%       3) The Copyright Line with ACM data
%       4) Page numbering
% ---------------------------------------------------------------------------------------------------------------
% It is an example which *does* use the .bib file (from which the .bbl file
% is produced).
% REMEMBER HOWEVER: After having produced the .bbl file,
% and prior to final submission,
% you need to 'insert'  your .bbl file into your source .tex file so as to provide
% ONE 'self-contained' source file.
%
% Questions regarding SIGS should be sent to
% Adrienne Griscti ---> griscti@acm.org
%
% Questions/suggestions regarding the guidelines, .tex and .cls files, etc. to
% Gerald Murray ---> murray@hq.acm.org
%
% For tracking purposes - this is V3.1SP - APRIL 2009

\documentclass[]{sigcomm-alternate}
\usepackage{url}
\usepackage[hidelinks]{hyperref}
\usepackage{multirow}
\usepackage[noend]{algpseudocode}
\usepackage{algorithm}
\usepackage{color}
\usepackage{verbatimbox}

\usepackage[pass]{geometry}
\setlength{\paperheight}{11in}
\setlength{\paperwidth}{8.5in}	

\newcommand{\ie}{i.e., \@}
\newcommand{\eg}{e.g., \@}
\newcommand{\Ie}{I.e., \@}
\newcommand{\Eg}{E.g., \@}

\renewcommand{\baselinestretch}{.990}

\newcommand{\todo}[1]{\textcolor{red}{TODO: \emph{#1}}}

\newcommand{\eat}[1]{}

\newenvironment{myitemize}
{
    \begin{list}{\labelitemi}{\leftmargin=1em}
        \setlength{\topsep}{0pt}
        \setlength{\parskip}{0pt}
        \setlength{\partopsep}{0pt}
        \setlength{\parsep}{0pt}         
        \setlength{\itemsep}{0pt} 
}
{
    \end{list} 
}

\begin{document}

\title{Predicting Adverse Thermal Events in a Smart Building}
%\titlenote{(Does NOT produce the permission block, copyright information nor page numbering). For use with ACM\_PROC\_ARTICLE-SP.CLS. Supported by ACM.}
%\subtitle{[Extended Abstract]
%\titlenote{A full version of this paper is available as
%\textit{Author's Guide to Preparing ACM SIG Proceedings Using
%\LaTeX$2_\epsilon$\ and BibTeX} at
%\texttt{www.acm.org/eaddress.htm}}}
%
% You need the command \numberofauthors to handle the 'placement
% and alignment' of the authors beneath the title.
%
% For aesthetic reasons, we recommend 'three authors at a time'
% i.e. three 'name/affiliation blocks' be placed beneath the title.
%
% NOTE: You are NOT restricted in how many 'rows' of
% "name/affiliations" may appear. We just ask that you restrict
% the number of 'columns' to three.
%
% Because of the available 'opening page real-estate'
% we ask you to refrain from putting more than six authors
% (two rows with three columns) beneath the article title.
% More than six makes the first-page appear very cluttered indeed.
%
% Use the \alignauthor commands to handle the names
% and affiliations for an 'aesthetic maximum' of six authors.
% Add names, affiliations, addresses for
% the seventh etc. author(s) as the argument for the
% \additionalauthors command.
% These 'additional authors' will be output/set for you
% without further effort on your part as the last section in
% the body of your article BEFORE References or any Appendices.

\numberofauthors{1}
\author{
\begin{tabular}{@{\extracolsep{\fill}}cc}
Jenna MacCarley & Federico Ponte\\
\affaddr{Carnegie Mellon University} & \affaddr{Carnegie Mellon University}\\
\email{jmaccarl@andrew.cmu.edu} & \email{federico.ponte@sv.cmu.edu} \\
\\
\\
Dan Yang & Victor Hu\\
\affaddr{Carnegie Mellon University} & \affaddr{Carnegie Mellon University}\\
\email{dan.yang@sv.cmu.edu} & \email{vyh@andrew.cmu.edu}
\end{tabular}\\
}

\maketitle


\begin{abstract}

\end{abstract}



\section{Introduction}\label{sec:intro}
\input{sections/intro}
\section{Related Work}\label{sec:related}
\input{sections/related}
\section{Algorithms}\label{sec:algo}
\input{sections/algo}
\section{Experiments}\label{sec:exp}
\input{sections/experiments}
\section{Conclusion and Future Work}\label{sec:conclusion}
\input{sections/conclusion}


% A category with the (minimum) three required fields
%\category{H.4}{Information Systems Applications}{Miscellaneous}
%A category including the fourth, optional field follows...
%\category{D.2.8}{Software Engineering}{Metrics}[complexity measures, performance measures]

%\terms{Theory}

%\keywords{ACM proceedings, \LaTeX, text tagging} % NOT required for Proceedings


%
% The following two commands are all you need in the
% initial runs of your .tex file to
% produce the bibliography for the citations in your paper.
\bibliographystyle{abbrv}
\bibliography{bibliography}  % sigproc.bib is the name of the Bibliography in this case
% You must have a proper ".bib" file
%  and remember to run:
% latex bibtex latex latex
% to resolve all references
%
% ACM needs 'a single self-contained file'!
%
%APPENDICES are optional
%\balancecolumns
%\appendix
%Appendix A

%\subsection{References}

\balancecolumns
% That's all folks!
\end{document}
